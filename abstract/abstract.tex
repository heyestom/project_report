\begin{abstract}

Blazon is the semi-formal language of family crest and heraldry, dating back to the twelfth century.  Using a well structured grammar to describe a coat of arms in a top down approach the language provides a robust yet flexible way to define a textual description of what is naturally a very graphical subject.  Using a test-driven development model I have produced a project which is capable of parsing a large subset of Blazon.  Encompassing several fields of Computer Science ranging from parsing to HTML5 graphics the application provides a platform that demonstrates how modern concepts and technologies can be used to represent a subject that pre-dates them by centuries. 


 Firstly giving a brief background and description of the language of Blazon, this report goes onto describes both how the Blazon parsing application works as well as and how the project was implemented and tested.  From a simple shield of a single colour through partitioning into sections and sub partitioning as well as covering different line-types before heading onto geometric charges, honourable charges and the rule of tincture before finally discussing semi-formal charges. 

 After thoroughly describing the language I go onto discuss the implementation of the project using test driven development.  Producing lots of iterations increasing the functionality of the project gradually and performing regression testing to ensure the soundness of my code base.  Initially starting with a couple of Python script based prototypes I describe how the project was iteratively built into a fully fledged web application over the course of the academic year.   Given more time I would attempt to expand the project further into parsing and drawing a larger subset of Blazon with features such as Counter Charging and combining shields with Quartering. 

\end{abstract}
