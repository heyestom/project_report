\chapter{Defining a Grammar for Blazon}

Now that the language of Blazon has been fully defined the task of Parsing it can begin.  Parsing a language is a fairly complicated task however there are several standard tools found throughout the field that simplify the problem by breaking it down into smaller more manageable tasks. The first goal is to express Blazon as a Grammar. 


\section{What is a Grammar?}
A grammar is best defined as a series of formal rules that define the syntax of a formal language.  Blazon is only a semi-formal language as charges provide components that are not predefined, however they are only a sub set of the language and can be handled as an edge case which can be included into the grammar.  

A sentence of a Language must conform to that Language's grammar.  To this end a grammar can be used to both produce sentences for a language but can also be used to validate that sentences conform to a Language. This is exactly what is needed for parsing. 


There are four parts in a grammar.  Firstly every grammar has a set of terminal or atomic symbols which are the building blocks of the language.  Secondly a grammar needs to have a set of non-terminal symbols which represent a series of terminal symbols.  Thirdly a grammar will have a series of production rules which will take a terminal symbol and turn it into a either terminal or non-terminal symbols or a combination of both. Finally a grammar must have a defined non-terminal symbol as a start symbol.





There are two types of grammar, context free and context sensitive the difference between the two is in the production rules.  A context sensitive grammar can have production rules where there are both terminal tokens and non-terminal tokens on both sides where context free grammars always uniquely have non-terminals on the left hand side of every production rule. 

\section{Context free Grammars}

Context free grammars 

$$ 	list \rightarrow  list + digit  $$
$$	list \to list - digit  $$
$$  list \to digit $$
$$  digit \to 0|1|2|3|4|5|6|7|8|9 $$



\section{Context Free Blazon}



\emph{Blazon as a context free grammer}
$$ 	Field \to  Partition + Field + Field  $$
$$	Field \to  Partition + Field + Field + Field  $$
$$	Field \to  Partition + Field + Field + Field + Field $$
$$  Field \to  Tinctured Field $$
$$  Tinctured Field \to Tinctured Field + Charge $$
$$  Charge \to Charge + Charge $$
$$  Charge \to An Ordinary  + A Tincture $$
$$  Charge \to A Geometric  + A Tincture $$
$$  Charge \to A semi-formal + A Tincture $$

%line types attitudes 

\section{Formatting Input}

\section{Tokenizing Strings} 