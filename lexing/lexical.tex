\chapter{Lexical Analysis} 


Before parsing a Lexical analysis must be performed upon the input. The aim of Lexical Analysis is present an input in a way that makes it easier to parse.  This involves taking an input and splitting it into a series of tokens using keywords as identifiers.


\section{What is Lexical Analysis?}
Lexical Analysis is more concerned with classifying sections of input as instances of things than it is understanding them; as such there are generally several kind of tokens produced by a lexical analysis, determined by the context of the application.  Tokens can range from being a single letter to being a series of words, again this is context dependant.

In traditional compilation all of the basic programming control statements would have their own Token; if, else, while, etc.  Tokens provide a basic abstraction level as in the sense that you can have a token representing an instance of a Tincture but the actual data about which Tincture it is an instance of is irrelevant at this stage, although it must be stored for later. 


\section{Lexical Analysis of Blazon} 


Lexically Analysing Blazon sentences is not 

With regard to Blazon the words to be Lexically analysed are potential Blazon sentences and Tinctures, Partitions, and Charges need to be classified.



  Fortunately in Blazon this is a fairly simple task.  A Partition is always prefixed \emph{"Per"} then followed by one of a pre-defined set of known partitions then an optional line type before a mandatory tincture. 


Charges are also always prefixed but with a \emph{Quantifier} and are either followed by a tincture or a line type and then tincture.  Whenever a Tincture occurs without being part of a Charge it has to be tincturing a field. 
 

