\chapter{Development}

The development of the application was performed using a test driven development model.  There were some initial prototypes thet encompassed a small subset of the Blazon language and gave a textual description only were developed also. 


\section{Background Research}
The first couple of weeks of development time were spent exclusively on researching Blazon.  The Blazon language provides a great number features  but is not particularly intuitive.  Two texts in particular were used \emph{ Heraldry, Its Origins and Meaning}\cite{bob} and \emph{The oxford guide to Heraldry}\cite{oxford}.  Both of the above provide a nice overview to heraldry with the former being very thorough on Blazon and the adoption of heraldry throughout Europe. 

\emph{Luke Torjussen}'s similar application\cite{luke} proved to be a very useful learning tool and later it was great for testing.  

Researching parsing techniques was also undertaken by consulting the \emph{Dragon Book}\cite{dragon}.  This text is considered the industry standard on parsing and compilers and provided a wealth of information. 


\section{Prototypes}

After researching the problem domain, three small prototypes were developed.  The first two were written in \emph{Python} and the third was a \emph{PHP} script.  

Each prototype was built to pass a series of pre-defined tests defining the scope of functionality.  This is how test driven development handles development life cycles; an iteration is treated as complete only when it passes all the tests set out for it. 

The prototypes really helped incorporate the idea iterative life cycles when using a test driven development model.  Each was a re-working of the iteration before  and encompassed a slightly larger subset of the Blazon language then the last.

The first \emph{Python} script incorporated the idea of Tinctures and and partitions, the second had a very naive implementation of geometric charges. The third prototype didn't increase the subset of the language being parsed but instead reflected an architectural change to the application as it was the first web based iteration. 



\section{Development of the Main Application}


The decision was taken to make the Blazon parser a stand alone client side web application, relying on no significant infrastructure other than a web host.  The reason behind this decision was to increase scalability and lower the workload of the server.  

After the prototyping phase, another build of the application was produced in \emph{JavaScript}, which is a web based client side scripting language supported by all major browsers.  This build was then developed iteratively over the course of several months.

Every iteration of the project increased the subset of Blazon that could be correctly parsed.



\section{Graphical Representation}

Once the parser was able to handle a large enough set of Blazon a graphical element was added to the project.  The idea was to provide an accurate graphical representation of the Blazon sentence parsed by the application.  

\emph{HTML5's Canvas} element was chosen as the tool for rendering images as it has wide browser support and can be controlled using \emph{JavaScript}.  The rendering speed of the Canvas element is imperceivable to the human eye even for moderately complex drawings.  

The development of the graphical side of the project was also iterative following a test driven model.  The iterations mirrored those of the parsing implementation in that they started with a small subset of Blazon, which was expanded on each subsequent iteration. 

